\documentclass[11pt]{article}
\usepackage[margin=1in]{geometry}
\usepackage[T1]{fontenc}
\usepackage[utf8]{inputenc}
\usepackage{lmodern}
\usepackage{amsmath,amssymb}
\usepackage{hyperref}

% pandoc compatibility
\providecommand{\tightlist}{%
  \setlength{\itemsep}{0pt}\setlength{\parskip}{0pt}}

% Unicode support for symbols present in the source
\DeclareUnicodeCharacter{2248}{\ensuremath{\approx}}
\DeclareUnicodeCharacter{2192}{\ensuremath{\rightarrow}}

% Prevent automatic section numbering (headings already include explicit numbers)
\setcounter{secnumdepth}{0}

\begin{document}

\begin{titlepage}
\thispagestyle{empty}
\raggedright

\vspace*{0.62\textheight}

{\Large\bfseries The Time-to-Analysis Layer\\\large Pressure Points in AI-Assisted Research
Systems\par}


\vspace{0.4em}
{\large\itshape AI research systems / product wedge\par}

\vspace{1.6em}

Tyson Jeffreys\\
Independent Researcher\\
\href{mailto:tyson@staygolden.dev}{tyson@staygolden.dev}

\vspace{1.6em}

Version 1.1 --- February 11, 2026

\end{titlepage}


\noindent\textbf{Series note (3/3).} This paper is Part 3 of a three-paper series on regime-level regulation in intelligent systems. Part 1 introduced two control regimes in embodied systems and a baseline regulator layer to reduce prolonged compensation. Part 2 proposed concept containers as representation-level regulation that stabilizes and reuses causal structure. Here we turn the same principle into a research-systems objective: reduce time-to-analysis with reusable analysis-layer artifacts and bursty synthesis.\par\medskip
\hypertarget{abstract}{%
\subsection{Abstract}\label{abstract}}

Most AI ``research assistants'' optimize retrieval and summarization. We
argue that the primary leverage point is different: minimizing
\textbf{time-to-analysis}---the latency between raw information and
\emph{intervention-ready understanding}. We define an \textbf{analysis
layer} as a system output that exposes causal structure, disagreement,
uncertainty, and decision levers, rather than producing flat summaries
or single answers.

We frame research as a \textbf{pressure point}: an upstream bottleneck
where modest improvements compound across downstream decisions. We then
show how analysis-layer outputs can reduce both human cognitive cost and
system compute cost by preventing repeated recomputation. Finally, we
outline an architecture for regulated research systems---low-cost
monitoring near baseline with brief, high-intensity synthesis
bursts---and propose falsifiable metrics and experiments.

\begin{center}\rule{0.5\linewidth}{0.5pt}\end{center}

\hypertarget{introduction}{%
\subsection{1. Introduction}\label{introduction}}

Human research workflows rarely fail because information is unavailable.
They fail because:

\begin{itemize}
\tightlist
\item
  relevant evidence is buried in noise
\item
  perspectives are fragmented across sources
\item
  contradictions are hard to see
\item
  the ``thinking'' phase is delayed by ingestion and context-switching
\end{itemize}

Current AI tools often accelerate \textbf{retrieval} and
\textbf{summarization}, but the user still must assemble structure:
causal models, levers, uncertainty, and what would change their mind.
The result is a familiar pattern: repeated reading, repeated synthesis,
repeated re-derivation.

This paper proposes a different objective for research tools:

\begin{quote}
\textbf{Optimize time-to-analysis, not time-to-text.}
\end{quote}

\begin{center}\rule{0.5\linewidth}{0.5pt}\end{center}

\hypertarget{pressure-points}{%
\subsection{2. Pressure points}\label{pressure-points}}

\hypertarget{definition}{%
\subsubsection{2.1 Definition}\label{definition}}

A \textbf{pressure point} is an upstream point in a system where modest
effort yields outsized downstream effects.

Pressure points have three properties:

\begin{enumerate}
\def\labelenumi{\arabic{enumi})}
\tightlist
\item
  \textbf{Upstream position}: affects many downstream actions
\item
  \textbf{Bottleneck}: currently constrains speed or quality
\item
  \textbf{Compounding}: improvements propagate multiplicatively
\end{enumerate}

\hypertarget{why-research-is-a-pressure-point}{%
\subsubsection{2.2 Why research is a pressure
point}\label{why-research-is-a-pressure-point}}

Research sits upstream of:

\begin{itemize}
\tightlist
\item
  strategy and prioritization
\item
  design and engineering decisions
\item
  safety and compliance judgments
\item
  belief formation and coordination
\end{itemize}

If time-to-analysis decreases, many downstream activities become faster
and more accurate---even if the downstream processes do not change.

\begin{center}\rule{0.5\linewidth}{0.5pt}\end{center}

\hypertarget{the-analysis-layer}{%
\subsection{3. The analysis layer}\label{the-analysis-layer}}

\hypertarget{definition-1}{%
\subsubsection{3.1 Definition}\label{definition-1}}

The \textbf{analysis layer} is the minimal output representation that
enables confident intervention decisions.

An analysis-layer output is not a single answer. It is a structured
object that includes:

\begin{itemize}
\tightlist
\item
  \textbf{Causal skeleton}: key variables and directional relations
\item
  \textbf{Disagreement map}: where sources or schools diverge
\item
  \textbf{Levers}: interventions that matter most
\item
  \textbf{Predictions}: what changes under each lever
\item
  \textbf{Falsifiers}: what evidence would overturn the model
\item
  \textbf{Uncertainty boundaries}: what is unknown vs contested
\end{itemize}

A useful way to state it:

\begin{quote}
The analysis layer turns ``information'' into a small set of
\emph{competing, testable causal stories}.
\end{quote}

\hypertarget{non-verifiable-selection-primitive}{%
\subsubsection{3.2 Non-verifiable selection
primitive}\label{non-verifiable-selection-primitive}}

Research, writing, and strategy typically lack a crisp verifier: there is no immediate ground truth to check a causal story against. In verifier-free domains, a common failure mode is unbounded synthesis---keep thinking until it feels right. Instead, treat \emph{selection} as an explicit primitive: generate a bounded set of candidate analysis artifacts, perform bounded comparisons (pairwise or tournament-style) using a critic/judge as a noisy \emph{sensor}, and either select a winner or abstain. Crucially, \textbf{tie/abstain mass} should be treated as a first-class uncertainty signal: high tie/abstain triggers evidence acquisition (more sources, better decomposition, new falsifiers) rather than further synthesis; low tie/abstain permits consolidation into a single intervention-ready artifact.\footnote{This pattern is aligned with verifier-free reasoning approaches that rely on demonstrations and pairwise comparisons, e.g., \emph{Escaping the Verifier: Learning to Reason via Demonstrations} (2025).}

\hypertarget{from-summaries-to-intervention-validity}{%
\subsubsection{3.3 From summaries to intervention
validity}\label{from-summaries-to-intervention-validity}}

A summary can be accurate and still useless for action. The analysis
layer is judged by a different standard:

\begin{itemize}
\tightlist
\item
  Can I identify what to do next?
\item
  Can I predict what will happen if I intervene?
\item
  Can I see what evidence would change the decision?
\end{itemize}

\begin{center}\rule{0.5\linewidth}{0.5pt}\end{center}

\hypertarget{time-to-analysis-metrics}{%
\subsection{4. Time-to-analysis
metrics}\label{time-to-analysis-metrics}}

\hypertarget{time-to-analysis-tta}{%
\subsubsection{4.1 Time-to-analysis (TTA)}\label{time-to-analysis-tta}}

Define \textbf{TTA} as the time required (human time + system time) to
produce an analysis-layer artifact that supports a stable decision.

In controlled tests, measure:

\begin{itemize}
\tightlist
\item
  wall-clock time to reach a decision
\item
  number of sources read
\item
  number of synthesis cycles
\item
  number of decision reversals after new evidence
\end{itemize}

\hypertarget{compute-to-analysis-cta}{%
\subsubsection{4.2 Compute-to-analysis
(CTA)}\label{compute-to-analysis-cta}}

For AI systems, define \textbf{CTA} as compute required to reach the
analysis layer:

\begin{itemize}
\tightlist
\item
  tokens generated
\item
  tool calls
\item
  retrieval operations
\item
  planner expansions / rollouts
\end{itemize}

A key claim of this paper is that analysis-layer systems can reduce CTA
by preventing repeated recomputation.

\hypertarget{decision-stability}{%
\subsubsection{4.3 Decision stability}\label{decision-stability}}

Define a \textbf{reversal} as a change in the selected decision or
causal model after incorporating additional evidence. A well-formed
analysis layer should reduce reversals because disagreements and
falsifiers are made explicit early.

\begin{center}\rule{0.5\linewidth}{0.5pt}\end{center}

\hypertarget{regulated-research-systems}{%
\subsection{5. Regulated research
systems}\label{regulated-research-systems}}

\hypertarget{baseline-vs-activation}{%
\subsubsection{5.1 Baseline vs
activation}\label{baseline-vs-activation}}

A research system should not operate at constant high synthesis.
Instead, it should behave like a regulated agent:

\begin{itemize}
\tightlist
\item
  remain in a low-cost monitoring baseline under low uncertainty
\item
  activate computation sharply when uncertainty, novelty, or stakes rise
\item
  collapse to baseline immediately after resolution
\end{itemize}

This is a posture claim: energy and robustness improve when activation
is episodic rather than continuous.

\hypertarget{analysis-bursts-and-container-formation}{%
\subsubsection{5.2 Analysis bursts and container
formation}\label{analysis-bursts-and-container-formation}}

A practical mechanism is to treat expensive synthesis as a
\textbf{burst} that produces a reusable artifact:

\begin{itemize}
\tightlist
\item
  build a causal structure from sources
\item
  output levers and falsifiers
\item
  store as a portable container with provenance
\end{itemize}

Future queries should reuse containers rather than re-deriving
structure.

\begin{center}\rule{0.5\linewidth}{0.5pt}\end{center}

\hypertarget{architecture}{%
\subsection{6. Architecture}\label{architecture}}

\hypertarget{pipeline-overview}{%
\subsubsection{6.1 Pipeline overview}\label{pipeline-overview}}

A minimal analysis-layer research system has five components:

\begin{enumerate}
\def\labelenumi{\arabic{enumi})}
\tightlist
\item
  \textbf{Evidence ingestion}

  \begin{itemize}
  \tightlist
  \item
    retrieve sources with coverage constraints
  \item
    track provenance and recency
  \end{itemize}
\item
  \textbf{Perspective decomposition}

  \begin{itemize}
  \tightlist
  \item
    cluster sources into schools / frames
  \item
    extract key claims and assumptions
  \end{itemize}
\item
  \textbf{Causal skeleton synthesis}

  \begin{itemize}
  \tightlist
  \item
    identify variables and relationships
  \item
    generate competing causal graphs or narratives
  \end{itemize}
\item
  \textbf{Lever + falsifier generation}

  \begin{itemize}
  \tightlist
  \item
    propose interventions that differentiate models
  \item
    propose evidence that would overturn each model
  \end{itemize}
\item
  \textbf{Container bank + reuse}

  \begin{itemize}
  \tightlist
  \item
    store analysis-layer artifacts
  \item
    retrieve and adapt for new contexts
  \end{itemize}
\end{enumerate}

\hypertarget{output-schema-analysis-artifact}{%
\subsubsection{6.2 Output schema (analysis
artifact)}\label{output-schema-analysis-artifact}}

A recommended output schema:

\begin{itemize}
\tightlist
\item
  \textbf{Handle}: short name
\item
  \textbf{Scope}: where it applies
\item
  \textbf{Causal skeleton}: variables + relations
\item
  \textbf{Disagreements}: competing claims + who holds them
\item
  \textbf{Levers}: actionable interventions
\item
  \textbf{Predictions}: outcomes under each lever
\item
  \textbf{Falsifiers}: evidence that breaks the model
\item
  \textbf{Uncertainty}: unknown vs contested
\item
  \textbf{Provenance}: sources + timestamps
\item
  \textbf{Confidence}: calibrated estimate
\end{itemize}

\hypertarget{guardrails-against-overconfidence}{%
\subsubsection{6.3 Guardrails against
overconfidence}\label{guardrails-against-overconfidence}}

Analysis-layer outputs should include:

\begin{itemize}
\tightlist
\item
  explicit unknowns
\item
  explicit falsifiers
\item
  provenance links
\item
  separation of ``consensus'' vs ``speculation''
\end{itemize}

\begin{center}\rule{0.5\linewidth}{0.5pt}\end{center}

\hypertarget{why-analysis-layers-reduce-compute-and-energy}{%
\subsection{7. Why analysis layers reduce compute and
energy}\label{why-analysis-layers-reduce-compute-and-energy}}

\hypertarget{avoiding-repeated-recomputation}{%
\subsubsection{7.1 Avoiding repeated
recomputation}\label{avoiding-repeated-recomputation}}

Many agent systems repeatedly:

\begin{itemize}
\tightlist
\item
  retrieve similar sources
\item
  re-summarize
\item
  re-synthesize the same causal model
\end{itemize}

Analysis-layer artifacts are designed to be reusable, so the expensive
part is amortized.

\hypertarget{duty-cycle-reduction}{%
\subsubsection{7.2 Duty-cycle reduction}\label{duty-cycle-reduction}}

If heavy synthesis is treated as an episodic burst, then the system's
high-compute duty cycle decreases:

\begin{itemize}
\tightlist
\item
  fewer long deliberation traces
\item
  fewer tool-call loops
\item
  fewer ``always-on'' monitoring cycles
\end{itemize}

This directly lowers compute-to-analysis and can improve
infrastructure-level efficiency.

\begin{center}\rule{0.5\linewidth}{0.5pt}\end{center}

\hypertarget{falsifiable-experiments}{%
\subsection{8. Falsifiable experiments}\label{falsifiable-experiments}}

\hypertarget{experiment-a-decision-tasks-with-controlled-evidence}{%
\subsubsection{8.1 Experiment A: decision tasks with controlled
evidence}\label{experiment-a-decision-tasks-with-controlled-evidence}}

Create tasks where participants must decide between interventions
(technical choice, policy choice, product strategy) using a set of
sources.

Conditions:

\begin{itemize}
\tightlist
\item
  A1: retrieval only
\item
  A2: retrieval + summary
\item
  A3: analysis layer (causal skeleton + disagreement + levers +
  falsifiers)
\end{itemize}

Measure:

\begin{itemize}
\tightlist
\item
  TTA (time-to-analysis)
\item
  decision accuracy (ground-truth or expert grading)
\item
  reversal count after additional evidence
\item
  CTA (tokens/tool calls)
\end{itemize}

Prediction:

\begin{itemize}
\tightlist
\item
  A3 reduces TTA and reversals at comparable or improved accuracy.
\end{itemize}

\hypertarget{experiment-b-transfer-robustness}{%
\subsubsection{8.2 Experiment B: transfer
robustness}\label{experiment-b-transfer-robustness}}

Hold causal structure constant while changing surface form (different
sources, different writing styles, different ordering).

Measure:

\begin{itemize}
\tightlist
\item
  stability of levers and falsifiers
\item
  CTA reduction via reuse
\end{itemize}

\hypertarget{experiment-c-adversarial-evidence-injection}{%
\subsubsection{8.3 Experiment C: adversarial evidence
injection}\label{experiment-c-adversarial-evidence-injection}}

Introduce conflicting or misleading sources mid-task.

Measure:

\begin{itemize}
\tightlist
\item
  whether the analysis layer flags uncertainty correctly
\item
  whether falsifiers trigger updates rather than thrash
\end{itemize}

Negative test:

\begin{itemize}
\tightlist
\item
  if analysis-layer outputs increase brittleness or overconfidence, the
  schema and calibration are insufficient.
\end{itemize}

\begin{center}\rule{0.5\linewidth}{0.5pt}\end{center}

\hypertarget{limitations}{%
\subsection{9. Limitations}\label{limitations}}

\begin{itemize}
\tightlist
\item
  \textbf{Domain dependence}: ``correct levers'' differ across domains.
\item
  \textbf{Calibration}: confidence estimates are hard; provenance helps
  but does not solve.
\item
  \textbf{Container drift}: stored artifacts must decay or be revisited
  as evidence changes.
\item
  \textbf{Incentive mismatch}: tools optimized for speed may hide
  uncertainty; analysis layers must surface it.
\end{itemize}

\begin{center}\rule{0.5\linewidth}{0.5pt}\end{center}

\hypertarget{conclusion}{%
\subsection{10. Conclusion}\label{conclusion}}

We proposed the time-to-analysis layer as a target for AI-assisted
research systems and argued that research is a strategic pressure point
where improvements compound across downstream decisions. An analysis
layer is defined by intervention-ready structure: causal skeletons,
disagreements, levers, falsifiers, and uncertainty boundaries.

The architectural claim is practical and falsifiable: \textbf{systems
that produce reusable analysis artifacts and regulate heavy synthesis as
episodic bursts should reduce compute-to-analysis, reduce decision
reversals, and improve time-to-decision at comparable accuracy}.

\begin{center}\rule{0.5\linewidth}{0.5pt}\end{center}

\hypertarget{appendix-a-suggested-figures}{%
\subsection{Appendix A: Suggested
figures}\label{appendix-a-suggested-figures}}

\begin{enumerate}
\def\labelenumi{\arabic{enumi})}
\tightlist
\item
  Retrieval vs summary vs analysis-layer output (three side-by-side
  boxes)
\item
  Time-series: high-compute bursts separated by low-cost baseline
  monitoring
\item
  Reuse diagram: analysis artifact produced once → reused across
  multiple downstream tasks
\end{enumerate}

\hypertarget{appendix-b-minimal-analysis-layer-checklist}{%
\subsection{Appendix B: Minimal ``analysis layer''
checklist}\label{appendix-b-minimal-analysis-layer-checklist}}

\begin{itemize}
\tightlist
\item
  What are the competing causal models?
\item
  Where do credible sources disagree?
\item
  What intervention would discriminate between models?
\item
  What evidence would change the conclusion?
\item
  What remains unknown (not merely uncertain)?
\end{itemize}
\newpage

{
\noindent\textbf{Note on authorship and tools:}\\
This work was developed through iterative reasoning, modeling, and synthesis.
Large language models were used as a collaborative tool to assist with drafting,
clarification, and cross-domain translation. All conceptual framing, structure,
and final judgments remain the responsibility of the author.
}

\end{document}
