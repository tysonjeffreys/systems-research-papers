\documentclass[11pt]{article}
\usepackage[margin=1in]{geometry}
\usepackage[T1]{fontenc}
\usepackage{lmodern}
\usepackage{microtype}
\usepackage{hyperref}
\usepackage{parskip}
\usepackage{csquotes}
\hypersetup{colorlinks=true,linkcolor=blue,urlcolor=blue,citecolor=blue}

\begin{document}

\begin{titlepage}
\thispagestyle{empty}
\raggedright

\vspace*{0.62\textheight}

{\LARGE Toward a Coherent Human Baseline\par}
\vspace{0.4cm}
{\Large Metabolic Regulation, Autonomic Quieting, and the Role of Meditation\par}

\vspace{1.6em}

Tyson Jeffreys\\
Independent Researcher\\
\href{mailto:tyson@staygolden.dev}{tyson@staygolden.dev}

\vspace{1.6em}

Version 1.0 --- January 5th, 2026

\end{titlepage}

\begin{abstract}
Recent work has proposed that modern humans operate above a historically typical metabolic and ventilatory baseline, characterized by chronic sympathetic activation and elevated respiratory drive. This paper extends that inquiry by describing a coherent physiological state reachable through breath regulation alone, independent of formal meditation practice. In this state, baseline ventilation decreases, autonomic tone shifts toward parasympathetic dominance, and perceptual and cognitive noise are reduced. Meditation is introduced as a secondary layer that further stabilizes and refines this baseline by quieting neural activity and promoting structural brain changes. Together, these observations describe a reproducible physiological condition and clarify the distinct role meditation may play in enhancing it.
\end{abstract}

\section{Introduction}
Most contemporary discussions of meditation and contemplative practice emphasize attention, cognition, or subjective experience. Less attention has been given to the baseline physiological conditions that precede and support these experiences.

In prior work, I proposed that modern humans exhibit chronically elevated metabolic and ventilatory activity relative to functional demand. This elevated baseline introduces persistent internal noise---physiological, emotional, and cognitive---that shapes perception and behavior. The present paper focuses on what occurs when that noise is systematically reduced.

Specifically, this paper documents a stable physiological state characterized by reduced ventilation, efficient gas exchange, and parasympathetic dominance. Importantly, this state is reachable without meditation through sustained breath regulation and diaphragm-led breathing. Meditation, when added, appears to further quiet neural activity and reinforce the stability of this baseline rather than create it.

\section{The Coherent Baseline State}

\subsection{Physiological characteristics}
The coherent baseline state described here exhibits the following features:
\begin{itemize}
\item Reduced respiratory rate and tidal volume relative to baseline
\item Stable oxygenation and carbon dioxide tolerance
\item Predominantly diaphragmatic breathing
\item Lower resting heart rate and increased heart rate variability
\item Absence of air hunger or compensatory breathing
\end{itemize}

This state is not one of sedation or passivity. Cognitive clarity, alertness, and responsiveness remain intact. Physical movement and exertion are still possible, with temporary sympathetic activation followed by rapid return to baseline.

\subsection{Autonomic tone}
In this state, parasympathetic influence predominates at rest. Emotional reactivity decreases, startle responses diminish, and the body exhibits a sense of internal steadiness. External stimuli are registered without triggering disproportionate physiological responses.

Subjectively, this often manifests as calmness or groundedness, though these descriptors are secondary effects rather than the primary phenomenon. The defining feature is reduced internal noise rather than the presence of a particular emotional state.

\section{Accessibility Without Meditation}
A central observation is that this baseline state does not require meditative training, attentional techniques, or altered states of consciousness. It can be established through consistent regulation of breathing patterns alone.

Sustained diaphragm-led breathing with reduced ventilation appears sufficient to recalibrate autonomic tone over time. The transition is gradual rather than immediate; in the observed case, adaptation unfolded over several weeks, during which the body progressively stabilized at the lower baseline without continuous conscious effort.

Once established, the state becomes largely self-maintaining. Breathing remains slow and quiet during normal cognition, speech, and light physical activity.

\section{Meditation as an Enhancement Layer}

\subsection{Distinguishing baseline from enhancement}
Meditation is often presented as the primary means of achieving calmness or clarity. The observations described here suggest a different ordering:
\begin{enumerate}
\item Physiological coherence establishes a low-noise baseline
\item Meditation further refines and stabilizes this condition
\end{enumerate}

In this framing, meditation operates most effectively when the body is already regulated. It does not initiate coherence but deepens it.

\subsection{Neural quieting and structural effects}
When practiced on top of a coherent physiological baseline, meditation is associated with:
\begin{itemize}
\item Further reduction in spontaneous thought activity
\item Increased stability of attention without effort
\item Diminished emotional reactivity
\item Subjective sense of mental spaciousness
\end{itemize}

Existing research associates long-term meditation with reduced amygdala reactivity and structural changes in cortical regions related to attention and regulation. Within the context described here, these changes can be interpreted as neural adaptations that reinforce an already quiet physiological environment.

Meditation, in this sense, functions as a stabilizing and amplifying layer rather than the foundational mechanism.

\section{Implications and Limitations}
The observations presented here are limited to a single-subject case and should be interpreted accordingly. However, the consistency and stability of the described state suggest that it reflects general properties of human physiology rather than idiosyncratic traits.

Potential directions for future investigation include:
\begin{itemize}
\item Objective metabolic and respiratory measurements across individuals
\item Time courses of adaptation to reduced ventilation
\item Interaction effects between baseline coherence and meditation depth
\end{itemize}

No claims are made regarding universal outcomes, exceptional abilities, or broader societal implications. The scope of this paper is limited to documenting a reproducible physiological condition and clarifying the role meditation may play in enhancing it.

\section{Conclusion}
A coherent human baseline characterized by reduced metabolic and ventilatory activity appears accessible through breath regulation alone. This state reduces internal physiological and cognitive noise, promotes parasympathetic dominance, and alters subjective experience in predictable ways.

Meditation, when layered onto this baseline, further quiets neural activity and stabilizes the condition but does not appear necessary to establish it. Recognizing this distinction clarifies the relationship between physiological regulation and contemplative practice.

Rather than representing an extraordinary achievement, this state may reflect a physiologically available baseline that has become uncommon in modern environments.

\clearpage
\thispagestyle{empty}
\begin{center}
{\Large Author’s Note}
\end{center}
\vspace{0.8cm}
This paper was authored by the undersigned. Large language model tools were used as a collaborative aid for drafting, editing, and clarity, while all concepts, observations, and conclusions remain the author’s own.

\end{document}
